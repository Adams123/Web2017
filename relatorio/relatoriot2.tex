\documentclass[10pt,a4paper]{article}

\usepackage{indentfirst}
\usepackage{amsthm,amsfonts,amsmath,amssymb}
\usepackage[brazilian]{babel}
\usepackage[T1]{fontenc}
\usepackage[utf8]{inputenc}
\usepackage{setspace}
\usepackage[usenames,dvipsnames]{xcolor} 
\usepackage{pgf,tikz}
\usepackage{float}
\usepackage{graphicx}
\usepackage{subfigure}
\usepackage{wrapfig}
\usepackage{multirow}
\usepackage{xcolor,colortbl}
\usepackage{changepage}
\usepackage{geometry}
\usepackage[pdftex]{hyperref}
\usepackage{listings}
\usepackage[normalem]{ulem}
\usepackage{enumitem}
\usepackage{booktabs}
\usepackage{multirow,array,varwidth}
\usepackage{tabularx}
\usepackage{makeidx}
\usepackage[nottoc]{tocbibind}
\usepackage{caption}
\usepackage{etoolbox}
\usepackage[pdftex]{hyperref}
\usepackage{longtable}
\usepackage{calc}
\geometry{a4paper,inner=2.0cm,outer=2.0cm,top=2.0cm,bottom=2.0cm}
\setenumerate[1]{label=\thesubsection.\arabic*.}
\setenumerate[2]{label*=\arabic*.}

\setlength{\tabcolsep}{6pt}

\newcommand*\NewPage{\newpage\null\thispagestyle{empty}\newpage}
\newcommand{\Barra}{\ensuremath{\backslash}}

\newcommand\Data[3]{\ensuremath{#1\backslash #2\backslash #3}}

\newcounter{magicrownumbers}
\newcommand\rownumber{\stepcounter{magicrownumbers}\arabic{magicrownumbers}}

\setlength{\columnsep}{1cm}
\addto\captionsbrazilian{% Replace "english" with the language you use
  \renewcommand{\contentsname}%
    {Tabela de Conteúdo}%
}
\hypersetup{
    colorlinks=true,
    linkcolor=blue,
    filecolor=magenta,      
    urlcolor=cyan,
}

\begin{document}

\thispagestyle{empty}
\begin{center}
	UNIVERSIDADE DE SÃO PAULO – USP
	
	INSTITUTO DE CIÊNCIAS MATEMÁTICAS E DE COMPUTAÇÃO
	
	DEPARTAMENTO DE SISTEMAS DE COMPUTAÇÃO
	
	\vspace{7cm}
	
	\Large{\textbf{Introdução a Sistemas WEB}}\\
	\small{\textbf{Projeto 2}}
	
	\vspace{6cm}
	
	Adams Vietro Codignotto da Silva - $6791943$ \\ 
	Antônio Pedro Lavezzo Mazzarolo - $8626232$ \\
	Gustavo Dutra Santana - $8532040$\\
	Veronica Vannini - $8517369$\\
	
	\vspace{6cm}
	
	São Carlos
	
	2017
\end{center}

\NewPage
\pagenumbering{arabic}

\tableofcontents

\newpage

\section{Descrição}
Ao abrir o site, há uma tela home e uma tela de login. Todos os dados são persistentes em uma base \textit{indexeddb}, então ao limpar a cache do navegador ou reiniciar o servidor, os dados são perdidos. O projeto contém um arquivo \textit{index.html}, onde é feita toda a interface, e um \textit{index.css} contendo as customizações. Os arquivos \textit{index.js} e \textit{initDB.js} contém, em sua maioria, funções de inicialização do servidor e chamadas iniciais de funções, como criação da bd, verificação se ela está populada, entre outras. Os arquivos .js contém funções referentes à manipulação da base respectiva. O arquivo \textit{generalDB.js} contém funções de uso comum entre as bases, e o \textit{dadosteste.js} contém os dados inicias usados para popular a base de dados, gerados por um gerador de dados. Existem 5 cadastros de cada tabela, exceto de serviços associados (existem 10) e de admin, que existe apenas um.

\section{Inicialização do sistema e dependências}
Foi utilizado um pequeno servidor em \textit{node.js} apenas para a execução do projeto, implementado no arquivo \textit{server.js}, então é necessário instalar o \textit{node.js} (link \href{https://nodejs.org/en/}{aqui}). Após isso, dentro da pasta \textbf{aplicação}, é necessário rodar o comando \textit{npm install} e \textit{npm start} em um terminal (windows ou linux). A página está rodando no endereço \textit{localhost:8080}. Caso a página não seja exibida, pode ser necessário usar o comando \textit{npm install connect serve-static} e novamente o \textit{npm start} (houveram casos que foi preciso e outros não).

\textbf{Todo o conteúdo do trabalho está disponível no \href{https://github.com/Adams123/Web2017}{github}.}

\section{Descrição das páginas}
Ao abrir o site, há uma barra de navegação superior contendo apenas duas abas: home e login.
\subsection{Home}
A aba home apenas apresenta uma mensagem de bem vindo, não importando em qual área ela se encontra (admin, cliente ou login).
\subsection{Login}
A aba login mostra um formulário para login, sendo que se o usuário e senha utilizados forem \textit{admin} e \textit{admin}, irá ser redirecionado para a página de administração. Este é o único \textit{admin} cadastrado. Temos 5 clientes pré cadastrados para testes.
\subsection{Área da administração}
\subsubsection{Produtos}
Ao clicar na aba Produtos, algumas opções são oferecidas ao administrador:
\begin{itemize}
\item \textbf{Adicionar produto:} ao clicar em \textbf{Adicionar produto}, é exibido um formulário contendo todas as informações necessárias para cadastro de produtos, de acordo com as especificações. Ao clicar em cadastrar, o produto é cadastrado de acordo com os campos preenchidos. A ID (código de barras) é escolhida pelo usuário, mas deve ser numérica e única. Todos os campos são obrigatórios.
\item \textbf{Atualizar produto:} o primeiro campo \textbf{ID} contém a chave de busca de qual produto será atualizado, e após clicar em pesquisar, os campos abaixo exibem os dados do produto daquela ID específica, e o administrador edita qualquer campo exceto a ID. Ao clicar em atualizar, os dados no servidor são atualizados. Todos os campos são obrigatórios.
\item \textbf{Apagar produto:} Funcionamento parecido com o \textbf{Atualizar produto}, com a diferença que nenhum campo é editável e o botão Remover exclui o produto referente aquela ID da base de dados.
\item \textbf{Consultar produto:} Uma listagem com todos os produtos cadastrados na base de dados, contendo, em ordem, a ID, foto, nome do produto, descrição, preço por unidade, quantidade em estoque e quantidade vendidos.
\end{itemize}
\subsubsection{Calendário}
Nesta aba há no canto superior esquerdo há um botão \textit{today} e duas setas. Ao clicar em qualquer um deles, um calendário com os serviços e horários cadastrados e disponíveis é exibido, e a navegação é feita utilizando as setas superiores. Não há iteração do admin com o calendário, mas ao passar o mouse por um horário reservado, uma \textit{tooltip} exibe os dados daquele horário, contendo o serviço e o pet alocado aquele horário.
\subsubsection{Usuários}
A aba usuários contém opções de cadastro de clientes e administradores. Ao clicar em qualquer uma das opções, um formulário contendo todas as informações necessárias para cada tipo de cadastro é exibido, e ao clicar em salvar, o formulário é enviado à base de dados e cadastrado sob sua respectiva categoria. Note que todos os admins são cadastrados com a senha padrão "admin" e os clientes são cadastrados com a senha padrão "cliente", cabendo a cada um alterar posteriormente sua senha.
\subsubsection{Serviços}
\begin{itemize}
\item \textbf{Cadastra servioço:} aqui temos o cadastro de serviços, pedindo as informações para o cadastro de um novo serviço. Todos os dados são necessários.
\item \textbf{Associar serviço a horário:} nesta aba há um formulário contendo as informações para cadastro de um novo serviço em um horário. Como precisamos de serviços e pets cadastrados, dois menus \textit{dropdown} oferecem as possíveis opções. Ao clicar em Salvar, o serviço seria adicionado à base de dados e atualizado no Calendário. É possível cadastrar mais de um serviço à um mesmo horário, uma vez que numa pet shop podem ser feitos vários serviços num mesmo horário.
\item \textbf{Liberar horário reservado: } nesta aba há um formulário contendo as informações para liberar serviço de um horário específico. Como precisamos de serviços e pets cadastrados, um menu \textit{dropdown} lista todos os serviços existentes, preenchendo a data e horário respectivos. Ao clicar em Salvar, o horário seria liberado e removido do Calendário.
\end{itemize}
\subsubsection{Profile}
\begin{itemize}
\item \textbf{Editar perfil:} Aqui é exibido um formulário preenchido com as informações já cadastradas do usuário. Note que o CPF, como é chave do perfil, não pode ser alterado. Ao alterar e salvar o formulário, as informações seriam atualizadas na base de dados e no perfil.
\item \textbf{Listar Ganhos:} Duas tabelas são exibidas: a de serviços e a de produtos. Na tabela de serviços, cada linha contém o nome do serviço e o lucro gerado por aquele respectivo serviço. Na tabela de produtos, cada linha contém a ID do produto e o valor ganho pelas vendas daquele produto. Na parte inferior, há o lucro total, somando todos os campos de lucro das duas tabelas,e os lucros totais por categoria.
\end{itemize}
\subsubsection{Logout}
Na seção de logout, apenas existe a confirmação de logout. Caso clique em não, o usuário é redirecionado à home. Caso contrário, é redirecionado à página inicial do site.
\newpage
\subsection{Área do cliente}
\subsubsection{Produtos}
Na aba Produtos, todos os produtos da loja são exibidos, com uma foto, nome e descrição, preço por unidade, quantidade disponível e um botão de Adicionar ao Carrinho a quantidade selecionada. A quantidade nunca irá exceder o número disponível em estoque daquele produto.
\subsubsection{Carrinho}
Aqui mostra todos os produtos adicionados ao carrinho, com as respectivas quantidades e valores unitários e valores totais na tabela. Abaixo, temos o total da compra, com as opções de comprar (onde o usuário seria levado à uma tela de pagamento, mas no caso apenas é deduzido da quantidade em estoque daquele produto a quantidade da compra do produto) ou esvaziar o carrinho, limpando a tabela e todas as operações feitas no carrinho de compras.
\subsubsection{Calendário}
Nesta aba há no canto superior direito há um botão \textit{today} e duas setas. Ao clicar em qualquer um deles, um calendário com os serviços e horários cadastrados e disponíveis é exibido, e a navegação é feita utilizando as setas. Não há iteração do cliente com o calendário, mas ao passar o mouse por um horário reservado, uma \textit{tooltip} exibe os dados daquele horário, contendo o serviço e o pet alocado aquele horário.
\subsubsection{Profile}
\begin{itemize}
\item \textbf{Cadastrar animais:} aqui há um formulário contendo as informações necessárias para cadastrar um novo pet, que será associado àquela conta. A id do animal é gerado pelo sistema.
\item \textbf{Listar animais:} é exibido uma lista de todos os pets associados à conta, com nome, raça, idade, foto, serviços associados ao pet, e o valor dos serviços daquele pet.
\item \textbf{Editar perfil:} Aqui é exibido um formulário preenchido com as informações já cadastradas do usuário. Ao alterar e salvar o formulário, as informações são atualizadas na base de dados. O CPF, como é chave do sistema, não pode ser alterado.
\end{itemize}
\subsubsection{Logout}
Na seção de logout, apenas existe a confirmação de logout. Caso clique em não, o usuário é redirecionado à home. Caso contrário, é redirecionado à página inicial do site.

\section{Testes}
Em sua maioria, foram feitos testes de consistência e de aceitação de valores nos campos, utilizando os dados de teste já inseridos e novos dados através do cadastro. Foi testado nos navegadores \textit{chrome} e \textit{firefox}. Em poucos casos, a \textit{indexeddb} mostrou problemas no \textit{firefox}, devido à funções não existentes no navegador. Tais bugs não foram corrigidos, pois algumas dessas funções, se trocadas, resultariam no refatoramento quase completo do código.
\end{document}