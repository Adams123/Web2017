\documentclass[10pt,a4paper]{article}

\usepackage{indentfirst}
\usepackage{amsthm,amsfonts,amsmath,amssymb}
\usepackage[brazilian]{babel}
\usepackage[T1]{fontenc}
\usepackage[utf8]{inputenc}
\usepackage{setspace}
\usepackage[usenames,dvipsnames]{xcolor} 
\usepackage{pgf,tikz}
\usepackage{float}
\usepackage{graphicx}
\usepackage{subfigure}
\usepackage{wrapfig}
\usepackage{multirow}
\usepackage{xcolor,colortbl}
\usepackage{changepage}
\usepackage{geometry}
\usepackage[pdftex]{hyperref}
\usepackage{listings}
\usepackage[normalem]{ulem}
\usepackage{enumitem}
\usepackage{booktabs}
\usepackage{multirow,array,varwidth}
\usepackage{tabularx}
\usepackage{makeidx}
\usepackage[nottoc]{tocbibind}
\usepackage{caption}
\usepackage{etoolbox}
\usepackage[pdftex]{hyperref}
\usepackage{longtable}
\usepackage{calc}
\geometry{a4paper,inner=2.0cm,outer=2.0cm,top=2.0cm,bottom=2.0cm}
\setenumerate[1]{label=\thesubsection.\arabic*.}
\setenumerate[2]{label*=\arabic*.}

\setlength{\tabcolsep}{6pt}

\newcommand*\NewPage{\newpage\null\thispagestyle{empty}\newpage}
\newcommand{\Barra}{\ensuremath{\backslash}}

\newcommand\Data[3]{\ensuremath{#1\backslash #2\backslash #3}}

\newcounter{magicrownumbers}
\newcommand\rownumber{\stepcounter{magicrownumbers}\arabic{magicrownumbers}}

\setlength{\columnsep}{1cm}
\addto\captionsbrazilian{% Replace "english" with the language you use
  \renewcommand{\contentsname}%
    {Tabela de Conteúdo}%
}
\hypersetup{
    colorlinks=true,
    linkcolor=blue,
    filecolor=magenta,      
    urlcolor=cyan,
}

\begin{document}

\thispagestyle{empty}
\begin{center}
	UNIVERSIDADE DE SÃO PAULO – USP
	
	INSTITUTO DE CIÊNCIAS MATEMÁTICAS E DE COMPUTAÇÃO
	
	DEPARTAMENTO DE SISTEMAS DE COMPUTAÇÃO
	
	\vspace{7cm}
	
	\Large{\textbf{Introdução a Sistemas WEB}}\\
	\small{\textbf{Projeto 1}}
	
	\vspace{6cm}
	
	Adams Vietro Codignotto da Silva - $6791943$ \\ 
	Antônio Pedro Lavezzo Mazzarolo - $8626232$ \\
	Gustavo Dutra Santana - $8532040$\\
	Veronica Vannini - $8517369$\\
	
	\vspace{6cm}
	
	São Carlos
	
	2017
\end{center}

\NewPage
\pagenumbering{arabic}

\tableofcontents

\newpage

\section{Descrição do Mock-up}

O Mock-up cobre todas os requisitos apresentados na descrição do projeto, e se tratando de um mock-up, algumas funcionalidades não foram implementadas (em maioria, no que se refere à envio e exibição de formulários e dados) pois envolveriam conexão com base de dados, autenticação e outras tarefas que não se encaixavam no propósito desta fase do projeto.

Ao abrir o site, há uma tela home e uma tela de login. A verificação se é um cliente ou admin é feita \textit{hardcoded}, assim como todos os dados exibidos, que estão \textit{hardcoded}. O projeto contém 3 arquivos principais: \textit{index.html, index.js} e \textit{index.css}. O \textit{index.html} contém toda a parte de visualização do site (o html puro). O \textit{index.js} contém alguns scripts javascript criados para melhorar a navegação e facilitar a implementação do código. já o \textit{index.css} contém toda a customização da página. Para o calendário foi utilizado um \textit{plugin} que facilita a criação e utilização do mesmo.

\section{Inicialização do Mock-up e dependências}
Foi utilizado um pequeno servidor em \textit{node.js} apenas para a execução do projeto, implementado no arquivo \textit{server.js}, então é necessário instalar o \textit{node.js} (link \href{https://nodejs.org/en/}{aqui}). Após isso, dentro da pasta \textbf{aplicação}, é necessário rodar o comando \textit{npm install} e \textit{npm start} em um terminal (windows ou linux). A página está rodando no endereço \textit{localhost:8080}. Caso a página não seja exibida, pode ser necessário usar o comando \textit{npm install connect serve-static} e novamente o \textit{npm start} (houveram casos que foi preciso e outros não).

\textbf{Todo o conteúdo do trabalho está disponível no \href{https://github.com/Adams123/Web2017}{github}.}

\section{Descrição das páginas}
Ao abrir o site, há uma barra de navegação superior contendo apenas duas abas: home e login.
\subsection{Home}
A aba home apenas apresenta uma mensagem de bem vindo, não importando em qual área ela se encontra (admin, cliente ou login).
\subsection{Login}
A aba login mostra um formulário para login, sendo que se o usuário e senha utilizados forem \textit{admin} e \textit{admin}, irá ser redirecionado para a página de administração. Qualquer outro caso será redirecionado para a página dos clientes.
\subsection{Área da administração}
\subsubsection{Produtos}
Ao clicar na aba Produtos, algumas opções são oferecidas ao administrador:
\begin{itemize}
\item \textbf{Adicionar produto:} ao clicar em \textbf{Adicionar produto}, é exibido um formulário contendo todas as informações necessárias para cadastro de produtos, de acordo com as especificações. Ao clicar em cadastrar, o produto seria cadastrado de acordo com os campos preenchidos. Note que, idealmente a ID não seria um campo, e sim gerado pelo sistema.
\item \textbf{Atualizar produto:} o primeiro campo \textbf{ID} contém a chave de busca de qual produto seria atualizado, e após clicar em pesquisar, os campos abaixo exibiriam os dados do produto daquela ID específica, e o administrador editaria todos os campos. Ao clicar em atualizar, os dados no servidor seriam atualizados.
\item \textbf{Apagar produto:} Funcionamento parecido com o \textbf{Atualizar produto}, com a diferença que nenhum campo será editável e o botão Remover excluirá o produto referente aquela ID da base de dados.
\item \textbf{Consultar produto:} Uma listagem com todos os produtos cadastrados na base de dados, contendo, em ordem, a ID, foto, nome do produto, descrição, preço por unidade, quantidade em estoque e quantidade vendidos
\end{itemize}
\subsubsection{Calendário}
Nesta aba há no canto superior esquerdo há um botão \textit{today} e duas setas. Ao clicar em qualquer um deles, um calendário com os serviços e horários cadastrados e disponíveis é exibido, e a navegação é feita utilizando as setas. Ao clicar em um horário disponível, uma caixa de diálogo mostra que o evento do clique foi detectado, exibindo a data e horário clicado. Funcionalmente, o admin seria levado à aba de serviços,  na área de \textit{Associar serviço a horário}, descrita mais a baixo. Ao clicar em um horário utilizado, é mostrado a foto do pet cadastrado no horário. Funcionalmente, o admin seria levado à aba de serviços,  na área de \textit{Liberar horário reservado}, descrita mais a baixo.
\subsubsection{Cadastrar}
A aba cadastrar contém opções de cadastro de clientes, administradores, produtos e serviços. Ao clicar em qualquer uma das opções, um formulário contendo todas as informações necessárias para cada tipo de cadastro é exibido, e ao clicar em salvar um formulário seria enviado à base de dados e cadastrado sob sua respectiva categoria.
\subsubsection{Serviços}
\begin{itemize}
\item \textbf{Associar serviço a horário:} nesta aba há um formulário contendo as informações para cadastro de um novo serviço em um horário. Caso o administrador tenha sido redirecionado da aba Calendário, as informações de data e horário já ficariam preenchidas. Ao clicar em Salvar, o serviço seria adicionado à base de dados e atualizado no Calendário.
\item \textbf{Liberar horário reservado: } nesta aba há um formulário contendo as informações para liberar serviço de um horário específico. Caso o administrador tenha sido redirecionado da aba Calendário, todas as informações já estariam preenchidas. Ao clicar em Salvar, o horário seria liberado e removido do Calendário.
\end{itemize}
\subsubsection{Profile}
\begin{itemize}
\item \textbf{Editar perfil:} Aqui é exibido um formulário preenchido com as informações já cadastradas do usuário. Ao alterar e salvar o formulário, as informações seriam atualizadas na base de dados e no perfil.
\item \textbf{Listar Ganhos:} Duas tabelas são exibidas: a de serviços e a de produtos. Na tabela de serviços, cada linha contém o nome do serviço e o lucro gerado por aquele respectivo serviço. Na tabela de produtos, cada linha contém a ID do produto e o valor ganho pelas vendas daquele produto. Na parte inferior, há o lucro total, somando todos os campos de lucro das duas tabelas.
\end{itemize}
\subsubsection{Logout}
Na seção de logout, apenas existe a confirmação de logout. Caso clique em não, o usuário é redirecionado à home. Caso contrário, é redirecionado à página inicial do site.
\newpage
\subsection{Área do cliente}
\subsubsection{Produtos}
Na aba Produtos, todos os produtos da loja são exibidos, com uma foto, nome e descrição, preço por unidade, quantidade disponível e um botão de Adicionar ao Carrinho.
\subsubsection{Carrinho}
Aqui mostra todos os produtos adicionados ao carrinho, com as respectivas quantidades e valores unitários e valores totais na tabela. A baixo, temos o total da compra, com as opções de comprar (onde o usuário seria levado à uma tela de pagamento) ou esvaziar o carrinho, limpando a tabela e todas as operações feitas no carrinho de compras.
\subsubsection{Calendário}
Nesta aba há no canto superior direito há um botão \textit{today} e duas setas. Ao clicar em qualquer um deles, um calendário com os serviços e horários cadastrados e disponíveis é exibido, e a navegação é feita utilizando as setas. Ao clicar em um horário disponível, é exibido uma tela contendo o serviço e o nome do pet a ser cadastrado naquele horário. Ao clicar em um horário ocupado, uma foto do pet para qual o horário está reservado é exibida.
\subsubsection{Profile}
\begin{itemize}
\item \textbf{Cadastrar animais:} aqui há um formulário contendo as informações necessárias para cadastrar um novo pet, que será associado àquela conta.
\item \textbf{Listar animais:} é exibido uma lista de todos os pets associados à conta, com nome, raça, idade, foto, serviços associados ao pet, e o valor dos serviços.
\item \textbf{Editar perfil:} Aqui é exibido um formulário preenchido com as informações já cadastradas do usuário. Ao alterar e salvar o formulário, as informações seriam atualizadas na base de dados e no perfil.
\end{itemize}
\subsubsection{Logout}
Na seção de logout, apenas existe a confirmação de logout. Caso clique em não, o usuário é redirecionado à home. Caso contrário, é redirecionado à página inicial do site.
\end{document}