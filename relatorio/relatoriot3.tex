\documentclass[10pt,a4paper]{article}

\usepackage{indentfirst}
\usepackage{amsthm,amsfonts,amsmath,amssymb}
\usepackage[brazilian]{babel}
\usepackage[T1]{fontenc}
\usepackage[utf8]{inputenc}
\usepackage{setspace}
\usepackage[usenames,dvipsnames]{xcolor} 
\usepackage{pgf,tikz}
\usepackage{float}
\usepackage{graphicx}
\usepackage{subfigure}
\usepackage{wrapfig}
\usepackage{multirow}
\usepackage{xcolor,colortbl}
\usepackage{changepage}
\usepackage{geometry}
\usepackage[pdftex]{hyperref}
\usepackage{listings}
\usepackage[normalem]{ulem}
\usepackage{enumitem}
\usepackage{booktabs}
\usepackage{multirow,array,varwidth}
\usepackage{tabularx}
\usepackage{makeidx}
\usepackage[nottoc]{tocbibind}
\usepackage{caption}
\usepackage{etoolbox}
\usepackage[pdftex]{hyperref}
\usepackage{longtable}
\usepackage{calc}
\geometry{a4paper,inner=2.0cm,outer=2.0cm,top=2.0cm,bottom=2.0cm}
\setenumerate[1]{label=\thesubsection.\arabic*.}
\setenumerate[2]{label*=\arabic*.}

\setlength{\tabcolsep}{6pt}

\newcommand*\NewPage{\newpage\null\thispagestyle{empty}\newpage}
\newcommand{\Barra}{\ensuremath{\backslash}}

\newcommand\Data[3]{\ensuremath{#1\backslash #2\backslash #3}}

\newcounter{magicrownumbers}
\newcommand\rownumber{\stepcounter{magicrownumbers}\arabic{magicrownumbers}}

\setlength{\columnsep}{1cm}
\addto\captionsbrazilian{% Replace "english" with the language you use
  \renewcommand{\contentsname}%
    {Tabela de Conteúdo}%
}
\hypersetup{
    colorlinks=true,
    linkcolor=blue,
    filecolor=magenta,      
    urlcolor=cyan,
}

\begin{document}

\thispagestyle{empty}
\begin{center}
	UNIVERSIDADE DE SÃO PAULO – USP
	
	INSTITUTO DE CIÊNCIAS MATEMÁTICAS E DE COMPUTAÇÃO
	
	DEPARTAMENTO DE SISTEMAS DE COMPUTAÇÃO
	
	\vspace{7cm}
	
	\Large{\textbf{Introdução a Sistemas WEB}}\\
	\small{\textbf{Projeto 3}}
	
	\vspace{6cm}
	
	Adams Vietro Codignotto da Silva - $6791943$ \\ 
	Antônio Pedro Lavezzo Mazzarolo - $8626232$ \\
	Gustavo Dutra Santana - $8532040$\\
	Veronica Vannini - $8517369$\\
	
	\vspace{6cm}
	
	São Carlos
	
	2017
\end{center}

\NewPage
\pagenumbering{arabic}

\tableofcontents

\newpage

\section{Descrição}
Ao abrir o site, há uma tela home, com as abas de login, produtos e serviços. É necessário ter o CouchDB instalado previamente. Foi utilizado a api React Routes para fluxo de dados e gerenciamento de páginas. Inicialmente há um script de criação da base, e para popular a mesma.

\section{Inicialização do sistema e dependências}
Foi utilizado um servidor em \textit{node.js} apenas para a execução do projeto, implementado no arquivo \textit{server.js}, então é necessário instalar o \textit{node.js} (link \href{https://nodejs.org/en/}{aqui}). Após isso, dentro da pasta \textbf{aplicação}, é necessário rodar o comando \textit{npm install} e \textit{npm start} em um terminal (windows ou linux). A página está rodando no endereço \textit{localhost:8080}.

\textbf{Todo o conteúdo do trabalho está disponível no \href{https://github.com/Adams123/Web2017}{github}.}

\section{Descrição das páginas}
Ao abrir o site, há uma barra de navegação superior contendo quatro abas: home, produtos, serviços e login.
\subsection{Home}
A aba home apenas apresenta uma mensagem de bem vindo, não importando em qual área ela se encontra (admin, cliente ou login).
\subsection{Produtos}
Na aba produtos, há um botão para inserir os dados novos, diretamente no CouchDB, e uma tabela exibindo todos os produtos existentes.
\subsection{Serviços}
Na aba serviços, há um formulário para inserção de novos serviços e uma tabela exibindo todo o conteúdo da tabela de serviços.
\subsection{Login}
A aba login mostra um formulário para login, sendo que se o usuário e senha utilizados forem \textit{admin} e \textit{admin}, irá ser redirecionado para a página de administração. Este é o único \textit{admin} cadastrado. Temos 5 clientes pré cadastrados para testes.
\section{Testes}
\end{document}